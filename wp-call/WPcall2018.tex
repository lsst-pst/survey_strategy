\documentclass[DM,toc,usenatbib]{lsstdoc}
%\documentclass[DM,lsstdraft,toc,usenatbib]{lsstdoc}

% Package imports go here
\usepackage{amsmath}	% Advanced maths commands
\usepackage{amssymb}
\usepackage{gensymb}  % degree symbol 
\usepackage{natbib}  % bibliography
\usepackage{cprotect} 
% Local commands go here

%% Journal abbreviations
%\bibliographystyle{aasjournal}

\title[Call for LSST Cadence White Papers]{Call for White Papers on \\ LSST Cadence Optimization} 

\author{\v{Z}eljko Ivezi\'{c}, Lynne Jones, Tiago Ribeiro, \\
                 the LSST Project Science Team, \\
                 and  the LSST Science Advisory Committee} 

\setDocRef{Document-28382}
\date{\today}
\setDocRevision{v2}
\setDocStatus{Released}
\setDocAbstract{%
The LSST community is invited to play a key role in the definition of LSST's Observing Strategy 
by submitting white papers to help refine the `main survey' and fully define the use of 10-20\% 
of time expected to be devoted to various `mini surveys' (including the `Deep Drilling 
mini surveys' and `Target of Opportunity' programs).
The LSST Science Requirements Document (SRD) places minimal constraints on the observing strategy, 
recognizing that science evolves and that the initial (by now more than a decade old) survey strategy 
would have to be redefined closer to first light. With LSST first light expected in 2020, and the LSST operations
phase starting in 2022, now is the time to undertake the final planning for the initial LSST observing strategy. While
the existing candidate baseline survey strategy has been primarily defined by the LSST Project, the final planning
must be undertaken hand-in-hand with the community. We seek science-driven input for observing strategy
properties such as the sky coverage, per-bandpass imaging depth, temporal coverage, and various detailed
observing constraints. There are no specific limitations on what kind of science programs will be considered, 
except that the four primary LSST science themes must remain the cornerstones of the LSST survey.  
This document is a solicitation for white papers to help plan these aspects of the LSST survey strategy. 

The deadline for white paper submission is November 30, 2018.
}

% Change history defined here. Will be inserted into
% correct place with \maketitle
% OLDEST FIRST: VERSION, DATE, DESCRIPTION, OWNER NAME
\setDocChangeRecord{%
\addtohist{1}{2018-05-16}{First internally circulated version.}{\v{Z}eljko Ivezi\'{c}}
\addtohist{2}{2018-06-30}{First released version.}{\v{Z}eljko Ivezi\'{c}}
}

\begin{document}

% Create the title page
% Table of contents will be added automatically if "toc" class option
% is used.
\maketitle


\section{Introduction} 

The Large Synoptic Survey Telescope (LSST) will provide an unprecedented optical 
imaging dataset that will support investigations of our Solar System, Galaxy and Universe, 
across half the celestial sphere and over ten years of repeated observation. LSST observations will be
scheduled automatically, with the scheduling algorithm designed to address all science goals 
and maximize observing efficiency for given observing constraints. Please see Appendix~\ref{sec:supp}
for a list of most relevant references describing LSST and its operations. LSST is constructing a 
flexible scheduling system that can respond to the unexpected and be re-optimized as the survey progresses.

Any implementation of LSST's 10-year observing strategy must meet the basic requirements described in the 
LSST Science Requirements Document (\href{http://ls.st/srd}{SRD}\footnote{The LSST Science 
Requirements Document (SRD) is available at \href{http://ls.st/srd}{http://ls.st/srd}})
for the core LSST science goals:
\begin{itemize}
\item constraining dark energy and dark matter,
\item taking an inventory of the Solar System,
\item exploring the transient optical sky, and
\item mapping the Milky Way.
\end{itemize}
However, in practice, the SRD intentionally places minimal quantitative constraints on the observing strategy,
primarily requiring:
\begin{itemize} 
\item A footprint for the `main survey' of at least 18,000 deg$^2$, which must be uniformly covered to 
a median of 825 30-second visits per 9.6 deg$^2$ field, summed over all six filters, $ugrizy$ (see SRD 
Tables 22 and 23). This places a minimum constraint on the time required to complete 
the main survey. Simulated surveys indicate that the main survey typically requires 85--90\% 
of the available time (10 years) to reach this benchmark; even with scheduling improvements, it is unlikely 
that the goals of the main survey could be met with a time allocation significantly below 80\%. 
\item Parallax and proper motion $1\sigma$ accuracies of 3~mas and 1~mas/yr per coordinate at $r=24$, 
respectively, in the main survey (see SRD Table 26), which places
a weak constraint on how visits are distributed throughout the lifetime of the survey and throughout a season.
\item Rapid revisits (40 seconds to 30 minutes) must be acquired over an area of at least 2000 deg$^2$ (see SRD table 25) for
very fast transient discovery; this requirement can usually be satisfied via simple field overlaps when surveying contiguous areas of sky. 
\end{itemize}
This leaves significant flexibility in the detailed cadence of observations within
the main survey footprint, including the distribution of visits within a year (or between seasons), the distribution between filters and 
the definition of a `visit' itself. Furthermore, these constraints apply to the main survey; the use of the 
remaining time (i.e., in mini surveys) is not constrained by the SRD.

A brief introduction to the baseline survey strategy, expanded background of the primary LSST science 
goals, and concise descriptions of how these goals drive the basic survey strategy and data processing
requirements are provided in the \href{http://ls.st/lop}{LSST Overview paper}\footnote{The LSST Overview 
paper is a living document available at \href{http://ls.st/lop}{http://ls.st/lop}.}. The Project has developed
a comprehensive survey operation simulations tool (`OpSim'), which has been used to develop
full survey simulations (for more details see Section 2.7.1 in the Overview paper and references listed
in Section C.1 here). 
This current candidate baseline survey strategy, as represented in the reference survey simulation (baseline2018a), 
includes the main survey and several candidate mini surveys (for more details, please see Appendix A): 
\begin{itemize}
\item The {\bf main ``wide-fast-deep'' survey}, which covers $\sim$18,000 deg$^2$ of sky within
the equatorial declination
range $-62^\circ < \delta < +2^\circ$, and excluding the central portion of the Galactic 
plane. Within the main survey, two visits\footnote{A `visit' here is an LSST default visit, which 
consists of two back-to-back 15 sec exposures, for a total of 30 sec of on-sky exposure time. These back-to-back exposures are always
in the same filter, separated only by the 2 second readout time.} 
per 9.6 deg$^2$ field (in either the same or different filters) are acquired in each night, to allow identification of moving objects and rapidly varying transients, and to improve
the reliability of the alert stream. These pairs of visits are repeated every three to four nights throughout the period the field is visible in each year (other nights are used to maximize the sky coverage). Each
field in the main survey receives about 825 visits throughout the ten years of the LSST survey, spread over the six LSST filters 
$ugrizy$. The quantitative SRD constraints on area coverage, number of visits, parallax and proper motion errors, and 
rapid-revisit rate (40 seconds -- 30 minutes) apply to visits obtained in the main survey. 
In the current reference simulation, the main survey uses 86.4\% of the available time.
\item The set of five {\bf Deep Drilling Field candidate mini surveys}, consisting of five specific field pointings for a total of $\sim$ 50 deg$^2$, 
which are observed with a much denser sampling rate. These mini surveys use a similar sequence of visits; the fields
are observed every three to four days, but in a sequence of multiple $grizy$ exposures during gray and bright time, and then
multiple sequential $u$ band exposures during dark time. The current deep drilling mini survey fields are aimed at extragalactic
science, providing a `gold sample' to calibrate the main survey, and to discover Type Ia supernovae. 
As implemented in the current reference simulation, the Deep Drilling Fields mini surveys use about 4.6\% of the available time
and the resulting co-added depth is typically about 1 magnitude deeper than the main ``wide-fast-deep'' survey. 
\item The {\bf Galactic Plane candidate mini survey} covers the central portion of the Galactic plane that is not included in the main survey, 
centered around $|l| = 0^\circ$ and covering $\sim$ 1860 deg$^2$.  It is observed at a much reduced rate compared to the main survey, 
and with a smaller total number of observations per field (30 visits per field and per filter, in $ugrizy$), so as to
provide astrometry and photometry of stars toward the Galactic center but without reaching the confusion limit in the coadded images.
There is no requirement for pairs of visits in each night in this area. As implemented in the current reference simulation, the Galactic Plane mini survey uses
about 1.6\% of the available time.
\item The {\bf North Ecliptic Spur candidate mini survey} covers the area north of $\delta = +2^\circ$ to $10^\circ$ north of the Ecliptic plane
and is intended to observe the entire Ecliptic plane for the purpose of inventorying the minor bodies in the Solar System. This area ($\sim$ 4160 deg$^2$) 
is observed on a schedule similar to the main survey, although with a smaller total number of visits per field and only in filters $griz$. 
As implemented in the current reference simulation, the North Ecliptic Spur mini survey uses about 5.5\% of the available time. 
\item The {\bf South Celestial Pole candidate mini survey} covers the region south of the main survey, to the South Celestial Pole, $\sim$ 2315 deg$^2$,
including the Magellanic Clouds. 
This mini survey is observed with a strategy similar to the Galactic Plane mini survey, with 30 visits per field per filter in $ugrizy$, 
and without requiring pairs of visits. This provides coverage of the Magellanic clouds, but without committing extensive time as these fields are
at high airmasses from the LSST site. As implemented in the current reference simulation, the South Celestial Pole mini survey uses
about 2.0\% of the available time.
\end{itemize}

The locations of four of the Deep Drilling mini survey fields, which are included among the five DD mini surveys in the current reference simulation, 
have been finalized and announced to the community\footnote{For details, including the field center coordinates, 
please see http://ls.st/bki} in 2012, while any remaining field locations 
have yet to be specified. The locations of these four deep fields\footnote{Note that three of these four fields are
closely spaced in Right Ascension: 0$^h$38$^m$, 2$^h$23$^m$, and 3$^h$33$^m$.} were 
the result of a community driven process 
with the goal of obtaining multi-wavelength coverage by ground and space-based facilities,  some of which may 
not exist at the start of the LSST survey\footnote{This approach indeed worked: a lot of additional Spitzer and XMM-Newton 
data were gathered!}.
These fields are intended to coincide with multi-wavelength surveys targeting the ELAIS-S1, XMM-LSS, Extended Chandra Deep Field-South, 
and COSMOS pointings. The cadence of observations within these (or any) deep drilling mini surveys is not finalized, although the coadded
depth may be expected to be at least one magnitude deeper than the LSST main survey in each filter.

Beyond the SRD constraints and commitment to the location of four deep drilling fields, the survey strategy as implemented in the candidate baseline 
above should not be considered guaranteed, and candidate mini surveys present in the current baseline may not be carried forward to
the initial operations survey strategy; for illustrative examples of some alternate survey strategies, please see Appendix~\ref{sec:surveys}.
All of the candidate mini surveys and the cadence in the main survey will be re-evaluated on the basis of community input in the next 
step of planning for the initial LSST survey strategy. Much of the existing survey strategy 
has been based on a mix of community and project input, but the overall balance has been project-driven. 
The detailed cadence of visits in the main survey, the detailed footprint for the main survey (beyond a minimum of 18,000 deg$^2$), 
the footprint for any other mini survey (except the location of the four pre-announced 
deep drilling fields), and the cadence or number of visits in any mini survey are open topics for optimization. 
The main purpose of this call for white papers is to solicit detailed input from the community interested in 
LSST science in order to design the optimum overall survey strategy. 


\subsection{Ongoing community feedback: The LSST Community Observing Strategy Evaluation Paper (COSEP)}

The LSST community is already playing a key role in the refinement of LSST's observing strategy 
by developing and analyzing metrics for quantifying the performance of simulated surveys generated by
the LSST project.  This work is being assembled in an open github community\footnote{
https://github.com/LSSTScienceCollaborations/ObservingStrategy}, 
in the form of a large cross-community survey evaluation paper
titled \href{http://ls.st/9fw}{`Science-Driven Optimization of the LSST Survey Strategy'}\footnote{The first 
version of this community observing strategy evaluation paper was published as arXiv:1708.04058 in August 2017.},
which will be referred to here as the Community Observing Strategy Evaluation Paper (COSEP). 
Chapter 1 and 2 of the COSEP provide a useful overview of the considerations involved in 
modifying the LSST survey strategy, as well as more details of the baseline survey strategy and 
examples of some possible variations in survey strategy, implemented in various simulated surveys.

The COSEP explores the effects of relatively small changes to the LSST survey strategy
on the detailed performance of the anticipated science investigations. The main lessons 
learned from the first (2017) version of this paper are: 
\begin{enumerate} 
\item The LSST Project should simulate, analyze and optimize the rolling cadence idea
(a non-uniform sampling in time in the wide-fast-deep survey designed to increase the frequency of 
observations for better coverage of variable phenomena on time scales of a few months, driven 
by supernovae, asteroids, and short-timescale stellar variability, at the cost of decreasing the 
frequency slightly on longer timescales), and 
\item The LSST Project should execute a systematic effort to further improve the ultimate
LSST survey strategy (e.g., sky coverage optimization, u band depth optimization, mini surveys). 
\end{enumerate} 
The baseline survey simulation used in the COSEP was carried out in 2016, using the version of
OpSim available then; the simulation software has been significantly improved and used to recently
produce baseline2018a, a new candidate baseline simulation. The two simulations are statistically consistent,
especially in terms of time spent on various mini surveys and their footprints. The main inprovements
in baseline2018a simulation include optimizations in observing closer to the meridian and better 
balancing of the rate of observations in the mini surveys over the survey lifetime.

Through the end of construction and commissioning, the COSEP will remain a living document 
to which the community continues to be welcome to contribute, as the main vehicle for the 
community to broadly communicate to the LSST Project regarding 
the scientific repercussions of various observing strategies. It will serve as a repository
of metrics, with performance evaluations of existing simulated surveys. Periodically, the LSST
Project Scientist, with support from the Project Science Team (PST) and the Science Advisory Committee (SAC), 
will review these metrics and evaluations, to fold updates into the survey optimization efforts.

While the COSEP will continue to provide the means to update evaluations of the survey strategy, the white 
papers solicited in this call are intended to provide the LSST science community with an opportunity 
to propose more significant, and detailed, modifications of the LSST survey strategy, including broad changes in
the mini surveys.

We anticipate that the performance evaluation components of the white papers solicited here will also 
be added to the COSEP, to provide a comprehensive reference point for survey strategy evaluation. 

\subsection{Motivation for this white paper call}

Guided by the community input summarized in the COSEP and further 
advice from the Science Advisory Committee (SAC), the LSST Construction Project has decided to
solicit detailed technical white papers for specific modifications of the current baseline survey strategy.

As discussed in more detail in Appendix A, analysis to date indicates that the baseline candidate
survey strategy, while meeting the basic science requirements for the LSST survey, can be meaningfully 
improved\footnote{For details, see Sections 1.1 
and 2.3 in the \href{http://ls.st/o5k}{COSEP}.}. The LSST \href{http://ls.st/srd}{SRD}
provides minimal constraints on the survey strategy details because it recognized that science evolves and that the 
initial, by now more than a decade old, survey strategy will have to be re-optimized closer to first 
light. With LSST first light expected in 2020, now is the time to undertake the final pre-commissioning
optimization\footnote{``Optimization'' used here does not imply its strict mathematical meaning. 
We are attempting to create the best survey strategy possible, but this is not a formal optimization 
effort of an objective function in part because the concept of an ``optimized'' survey depends in part on a subjective
weighting of different science goals.} 
of the LSST baseline observing strategy. We seek science-driven input for cadence 
properties such as per-bandpass imaging depth, the sky coverage, temporal coverage, observing
rules, etc., as summarized in Appendix A. Investigations of a limited number of such survey strategy 
modifications are reported in Chapter 2 from the COSEP (and discussed 
in various supplementary materials listed in Appendix C). 


\subsection{General guidelines} 

We solicit detailed white papers for specific modifications of the current baseline survey strategy, including 
both the main survey and the deep drilling fields and other mini surveys. There are no 
specific limitations on what kind of science programs will be considered, but please note that 
the primary four LSST science themes remain the cornerstones of the LSST survey and must be
maintained. Indeed, the LSST SRD states that ``the adopted observing 
strategy will not jeopardize the goals of any of the four main science themes''. Each white paper should
include detailed metrics to evaluate the scientific performance and impact of the suggested changes; these metrics
do not need to be comprehensive across all science use cases\footnote{The LSST Project will provide metrics to evaluate
the SRD-required quantities, as well as run a combination of metrics contributed by the general science community
(as described in the COSEP) to cover a wide range of science goals.}. Descriptions of these metrics should be 
sufficiently detailed for the Project team
to be able to implement them. While the code for computing metrics is not required, Python 3 implementations
will be welcome. 

The detailed cadence of visits in the main survey, the footprint for any other mini survey (except the location of the four pre-announced
deep drilling fields), and the cadence or number of visits in any mini survey are open topics for optimization. The existence of a 
particular survey strategy for a given mini survey in the current baseline should not be taken to imply that it will be present in the same way in the final 
survey strategy. Science justification and accompanying scientific performance metrics that support a given mini survey are highly encouraged.

Detailed white papers are also solicited for novel ideas, such as twilight observing (see Appendix~\ref{sec:twilight}), 
as well as changes to the survey strategy that would create synergies with other major surveys ({\it e.g.}, WFIRST, Euclid). 
In addition, white papers describing even very small programs are welcome and encouraged. 
White papers should emphasize programs leveraging LSST's unique large \'{e}tendue.  
At this time we are not soliciting proposals to optimize observations during the commissioning period.

Technical constraints imposed by the system and observing conditions are summarized in 
Appendix B. Links to the description of LSST data processing and pipelines are included in Appendix~\ref{append:supplemental}. If a proposed dataset will require special processing beyond 
what standard LSST software will provide, a plan to obtain necessary software and computational resources 
must be provided in the white paper. In cases that require more detail, or in case of specific questions not 
addressed in this document, please start a discussion at \href{http://community.lsst.org}{community.lsst.org}\footnote{\url{https://community.lsst.org/c/sci/survey-strategy}}.

The LSST Science Requirements Document states that ``the adopted baseline design assumes a 
nominal 10-year duration with about 90\% of the observing time allocated for the main LSST survey'',
and further clarifies that ``The remaining 10\% of observing time will be used to obtain improved 
coverage of parameter space\dots''. While the detailed time allocation will eventually depend on currently unknown system
performance parameters, it is unlikely that the goals of the main survey could be met with a time allocation
significantly below 80\%. In the current baseline reference survey simulation, the main survey requires 86\% of the total time.
In other words, it is plausible that the time allocated to programs other
than the main survey could significantly exceed 10\% (perhaps by as much as a factor of two), but 
no firm commitments beyond this statement of plausibility can be made at this time. 

The data from any given specialized survey will be treated in exactly the same way as all LSST 
data: there is no special proprietary access or ownership for any subset of the data from LSST. The final set of 
deep-drilling fields and other mini surveys are likely to be based on an amalgam of ideas from different 
white papers; a survey strategy described in a given white paper need not be accepted or rejected without adaptation.


\subsection{Review process and timeline}

The deadline for submitting white papers is November 30, 2018. For submission instructions, 
please see the next section. 

Soon after the November 2018 submission deadline, members of the LSST Science Advisory Committee (SAC), 
with technical support from the Project, 
will undertake an initial review and decide which submitted white papers meet the criteria of scientific excellence and 
technical feasibility for further analysis. These criteria are discussed in Section~\ref{sec:ranking}. The SAC will publish 
a short summary of the results of this initial review, including a list of all white papers to be considered for further 
simulations, by April 2019. 

The input from the submitted white papers will be used to design multiple
options in observing strategies, and the Project team will generate a series of
simulations based on these survey strategies. These multiple simulated surveys will address 
varying science drivers and will form a ``menu'' of possible survey strategies (e.g., a main 
survey with 18,000 deg$^2$ of sky coverage vs. a main survey with sky coverage of 23,000 deg$^2$, perhaps to a
somewhat shallower depth). The performance evaluation criteria submitted as part of the 
white papers and in the COSEP will be used to generate quantitative assessments to compare these strategies.
During this process, the Project team will keep a public list of simulations in progress and notify authors
of the submitted white papers as to how their suggested survey strategy has been implemented
in these simulated surveys or combined with other suggested survey strategies.
We will also publish monthly updates on the \href{https://community.lsst.org/c/sci}{LSST Community}\footnote{\url{http://community.lsst.org/c/sci/survey-strategy}}
forum, an open and searchable forum platform for communication between the 
LSST Project and the astronomical community.

We anticipate that the list of observing strategies that will be simulated and analyzed 
(the ``menu'' above) will be available by April 2019. Simulated survey outputs, generated via the LSST 
Operations Simulator (OpSim) and the LSST Metric Analysis 
Framework (MAF) analyses will become available by the end 
of 2019 (see Appendix~\ref{append:supplemental} for more information about OpSim and MAF). 

A Survey Strategy Committee (SSC) will be established by the LSST Operations Director 
in 2020. An advisory report on the performance of the survey strategies developed as a result of this call for white papers, based on 
performance metrics provided in the white papers and the COSEP, will be prepared by the Project for this 
committee in early 2020. The SSC will undertake the work of balancing the overall scientific performance between the 
individual science goals, and will advise the Project on the specific survey strategy to be
used at the start of full LSST operations, as well as on guidelines for the strategy throughout all ten years of the survey. 
In developing their recommendations to the Director, the SSC will be guided by selection 
criteria set by the Project Science Team and discussed in Section~\ref{sec:ranking} ({\it e.g.}, restrictions 
based on technical criteria, such as those discussed in Appendix B). The Director can further consult with the Project Science 
Team about the SSC survey strategy recommendations. A baseline simulation that reproduces the
adopted strategy, and its detailed performance analysis, will be published in 2021. 
The start of LSST operations is anticipated in 2022. The SSC will continue to regularly explore 
cadence and survey progress throughout the lifetime of survey. It is likely that the observing
strategy will be further refined as the LSST system becomes better understood, and science goals 
further evolve.

An overall aim of the Project and all stakeholders is to make this process transparent to the community and to base 
all decisions on quantitative input and pre-defined criteria to the maximum extent possible. 
The Project will organize a dedicated session at the LSST 2018 Project and Community Workshop (Tucson, Aug 13-17)
about this call for white papers, to further clarify details, exchange ideas, discuss simulated surveys, 
and coordinate teams that plan to submit white papers. 
In addition, LSST Corporation and the Simons Center for Computational Astrophysics
will organize an LSST Cadence Hackathon at the Flatiron Institute in
Manhattan, on September 17 and 18, 2018, with an optional third day on September 19.  
At the Hackathon, particiants from different LSST Science
Collaborations will work together in small teams to explore creative
cadence strategies with the help from Project experts for  the OpSim and MAF
software packages. Members of teams that come up with the most
promising strategies will each receive a modest honorarium for writing
up their ideas as white papers. There is no registration fee for this workshop and 
travel support will be provided for attendees from LSSTC member
institutions, and for other attendees as funds allow.
To apply for the workshop, please fill out the application form at http://ls.st/cwg
by July 20, 2018. 


\begin{table}[htp]
\caption{\bf{Timeline for survey strategy work}}
\begin{center}
\begin{tabular}{l|l}
Call for white papers & June 30, 2018 \\
2018 Project and Community Workshop & Aug 13-17, 2018 \\
White paper submission deadline & Nov 30, 2018 \\
List of to-be-simulated survey strategies & April 2019 \\
Simulated survey strategies available & End of 2019 \\
Survey Strategy Committee (SSC) established & Early 2020 \\
Advisory report from Project to SSC & Early 2020 \\
SSC report on official initial LSST survey strategy & Early 2021 \\
Baseline simulation of initial LSST survey strategy & Mid 2021 \\
Start of LSST Operations & 2022 \\
Regular survey reviews by the SSC & 2022-2032 \\
\end{tabular}
\end{center}
\end{table}



\newpage
\section{White paper submission guidelines} \label{sec:guidelines}


\subsection{Who can submit a white paper?} 

All members of the scientific community interested in LSST science\footnote{Note that there is no 
requirement to have LSST data rights to submit a white paper.} can submit a white paper.
We encourage the LSST Science Collaborations to submit white papers resulting from collaborative
work among their members.
We reiterate that the data from any given specialized survey will be treated exactly the same 
way as all LSST data: the authors of white papers will have no proprietary access to it. 

\subsection{Requested input \label{sec:reqinput}}

The current candidate baseline survey strategy (as represented in the reference survey simulation baseline2018a) 
includes:
\begin{enumerate}
\item {\bf Committed Surveys} (white papers should be submitted to optimize the design of these):
\begin{itemize}
\item the main Wide-Fast-Deep (WFD) survey,
\item four of the Deep Drilling (DD) mini surveys,
\end{itemize}
\item {\bf Candidate Mini Surveys} (must be proposed and selected in white paper process):
\begin{itemize}
\item additional Deep Drilling (DD) mini surveys,
\item the candidate Galactic Plane (GP) mini survey,
\item the candidate Northern Ecliptic Spur (NES) mini survey, and
\item the candidate South Celestial Pole (SCP) mini survey. 
\end{itemize}
\end{enumerate}

Quantitative, science-driven optimization input is requested for each of the above, as well as 
for survey strategy questions such as optimal visit exposure time,
co-added per-bandpass imaging depth, the sky coverage, temporal coverage, observing
rules, etc.  For a detailed discussion, please see Appendix A. The existence of a particular
mini survey, sequence of observations, or exposure time in the current baseline does not 
guarantee its existence in the same or any form in the final survey strategy. 

In addition to existing surveys listed above, we also seek input on new mini survey ideas 
to replace, supplement, and/or enhance the current programs, including special ``Target of Opportunity'' (ToO) 
programs. We also solicit novel ideas, such as twilight observing (see Appendix~\ref{sec:twilight}),
as well as observing synergies with other major surveys (e.g., WFIRST, Euclid). 

We reiterate that the observing time allocated to {\bf all} the mini surveys 
(including all DD mini surveys) is ``about'' 10\%, per SRD design, and is unlikely to exceed 20\% of the total available
observing time; in the current candidate baseline, the mini surveys use about 14\% of the available time. 
We note that implementation details for mini surveys (especially DD mini surveys)
are coupled at some level to the main survey: more time for the former means less 
for the latter, and some of the design decisions for the latter affect the science
case for the former.  For example, some rolling cadence strategies for the main survey  
may allow some variable and transient science to happen that would otherwise be the 
focus of a deep drilling field, and changes in the main survey footprint will affect the 
definition of a Galactic Plane survey. The Project will provide additional alternative 
survey simulations with this call for white papers (please see the list in Appendix~\ref{sec:surveys}) 
to help illustrate some of these potential crossovers. 

\subsection{TeX template for submission \label{sec:textemplate}} 

Each proposed modification of the survey strategy must contain a Scientific Motivation, Technical Description, and Performance Evaluation section, instructions for which are expanded in the white paper submission template. The Scientific Motivation section is intended to explain why this survey strategy modification is important and what could be learned if the proposed observations were obtained, relative to some baseline. The Technical Description section details what observations are being requested and should provide enough detail to enable proper simulations\footnote{For more information on how survey simulations are created, please see the documentation on OpSim at \href{https://lsst-sims.github.io/sims_ocs/}{https://lsst-sims.github.io/sims\_ocs/}.} to be created, as well as additional information that can help in the process of combining similar but separate survey strategy modification requests. 

The Performance Evaluation section must contain methods to evaluate the effectiveness of the survey strategy modifications, particularly in light of the potential effect of survey strategy changes proposed in other white papers. It is unlikely that any proposed cadence modification suggested in white papers will be carried out in its original form; thus metrics (along with threshold values for these figures of merit) to evaluate the science performance of non-ideal simulations are crucial.  Note that each white paper does not need to solve or address the global optimization problem, nor does it need to supply performance criteria across all science goals\footnote{Note that 
the Project team cannot support individuals or groups wishing to run the Operations Simulator themselves. It is not required that authors of white papers run the proposed simulations themselves.}. The TeX submission template gives further instructions for each of these areas.

The submission template and an example of the submission, with instructions about
all the required information, can be found in the git repository hosted at \href{https://github.com/lsst-pst/survey_strategy_wp}{lsst-pst/survey\_strategy\_wp}\footnote{\url{https://github.com/lsst-pst/survey_strategy_wp}}.
To submit white papers, please email the compiled PDF to \href{lsst-survey-strategy@lists.lsst.org}{lsst-survey-strategy@lists.lsst.org}. An acknowledgement of the receipt of each white paper will be returned within 48 hours.

The Project will organize a dedicated session at the LSST 2019 Project and Community Workshop
(mostly likely during August 2019 in Tucson) to discuss submitted white papers and follow-up
deliberations and work. LSST Corporation is committed to provide funds to help waive registration
fee for this meeting for one person per submitted white paper (or at least a fraction of the registration 
fee in case of a large number of submitted white papers). 

For additional help or questions, please ask on \href{https://community.lsst.org/c/sci}{LSST Community}\footnote{\url{http://community.lsst.org/c/sci/survey-strategy}}. 


\section{White paper ranking criteria \label{sec:ranking}} 

The LSST observing strategy will aim to deliver a cutting-edge data set to enable
the four cornerstone scientific programs, while at the same time maximizing the 
science possible with specialized observing modes using about 10\% of the total observing time. 

We anticipate that the adopted observing strategy will be based on an amalgam of ideas from 
different white papers; therefore, there will be no formal ``acceptance'' or ``rejection'' of
proposed ideas. The overall ranking priority advice provided to the Project by the Science Advisory Committee 
will be based on the following considerations: 
\begin{itemize}
\item {\bf Science} Importance and legacy value of the proposed science program, including 
           its match to the unique abilities of the LSST system, and its consistency with the 
           four main LSST science themes. This could be a single strong science goal, synergies with
           other astronomical facilities, or a multitude of science goals enabled by the same survey strategy. 
\item {\bf Feasibility} Programs must be feasible from the hardware and software point of view (see Appendix B),
         specifically including any special data processing required. The complexity of the program, in terms of additional
         requirements on hardware and software beyond the baseline requirements, will also be considered. 
         The Project Science Team will conduct a technical check on submitted white papers.
\item {\bf Time requested} The amount of time (including overheads) required should be justified by 
        the associated science. 
\end{itemize} 

As always, the science program properties such as importance and robustness are open
to interpretation. It is inevitable that some science drivers will be in conflict, and even
observing efficiency may not be defined in an absolute sense. The Project, the Science Advisory 
Committee and the Survey Strategy Committee will strive to make the white paper ranking process 
transparent to the maximum extent possible. 

\vskip 0.2in 
{\it Acknowledgments:} this document has greatly benefited from discussions between 
the LSST Project Science Team, the LSST Science Advisory Committee and Kem Cook, 
Phil Marshall, Steve Ridgway, Daniel Rothchild, Peter Yoachim and numerous other members 
of the LSST Science Collaborations. 

\newpage
\appendix
\section{Examples of open survey strategy optimization questions \label{sec:optimization}} 

The quantitative optimization of the LSST observing strategy requires many 
detailed decisions to be made, often with only an indirect science justification,
or with conflicting science drivers.  The current most significant open questions and associated 
tradeoffs are listed below for the main survey and each mini survey. White papers are specifically 
encouraged to address these questions.

\begin{figure}[htb]
\centering
\includegraphics[width=0.8\textwidth]{Nvisits_all}
\caption{The current baseline survey includes the main Wide-Fast-Deep survey and eight candidate mini surveys:
the North Ecliptic Spur, the Galactic Plane, the South Celestial Pole, and the five fields of Deep Drilling mini surveys.
This figure demonstrates their footprint in the current baseline simulated survey and the total number of visits in all bands.
We are seeking suggestions for modification of the survey strategy, especially suggestions for changes to the
mini surveys.}
\end{figure}

\subsection{The main Wide-Fast-Deep survey} 

The baseline survey strategy optimizes the amount of sky covered in any given night (subject to 
the constraint of gathering pairs of visits in each night - generally, but not always, in a single filter), 
and allows the entire sky visible at any time of the year to be covered in about three nights. 
The basic strategy is designed to give roughly uniform coverage over the sky at any given time, and to reach
the survey goals for measuring stellar parallax and proper motion, and the number of visits per 
field (825 visits, summed over the six filters). In the baseline implementation, the main survey 
covers about 18,000 deg$^2$ of high Galactic latitude sky, and uses about 86\% of the available 
observing time (based on current survey simulations). 

Open questions and optimization options associated with the main survey are as follows: 
\begin{itemize}
\item With the current declination boundaries at $\delta = -62^\circ$ and $\delta = +2^\circ$,
the main survey area includes about 18,000 deg$^2$ (without the Galactic plane 
confusion zone, see Appendix~\ref{sec:GP}). These boundaries were set to optimize the number of
detected galaxies useful for cosmological studies, including weak lensing, with the declination
limits defined by an airmass limit of 1.18 (the maximum value of the minimum achievable airmass;
the maximum allowed observation airmass is set to 1.4). These galaxy counts 
stay within 5-10\% of the current baseline values even with a much larger survey area. 
For example, Section 2.4 in the COSEP describes a simulated survey
that covers 27,400 deg$^2$ to about 0.15 mag shallower co-added depth than in the baseline 
survey (the declination boundaries are at $\delta = -78^\circ$ and $\delta = +18^\circ$, relaxing the 
airmass limit to 1.5, and resulting in a mean number of visits per field about 20\% smaller that
in the baseline survey). This tradeoff between the sky coverage and number of visits is still
open to optimization, subject to the SRD constraints of a minimum 18,000 deg$^2$ footprint with at least 825 
visits per field. It is important to receive science-driven optimization arguments
from all science programs.  
\item The observing time allocation per band (number of visits per filter) listed in the Science Requirements Document
(Table 24) is given only as an illustration. Further optimization of this allocation for the main 
survey, and different fractional allocations for mini surveys, are possible. 
\item Concentrating a fraction of the observations for a given field into a shorter period of time
(a.k.a. a ``rolling cadence'') can provide enhanced
sampling rates over a part of the survey for a designated time, at the
cost of reduced sampling rate the rest of the time (while maintaining the nominal total 
visit counts). This would mean that LSST would not simply aim to cover all the visible sky every three nights,
as the current baseline cadence does. 
While it is likely that science programs such as supernovae, asteroids, and
short-timescale stellar variability would benefit from rolling cadence, detailed cadence
parameters have not been optimized yet ({\it e.g.}, how much of the survey area to ``roll'' at once 
and how long to ``roll'' for, or whether to ``roll'' in right ascension or declination).
The optimization of ``rolling cadence'' simulations will be driven by submitted white papers. 
\item The current baseline survey strategy obtains two visits per night (within 15-60 minutes) in 
order to enable easy linking of asteroid detections, and robust identification of rapid 
photometric transients. Whether the two visits on the same night should be obtained 
in the same filter or in different filters has not been decided yet (e.g., in the context
of photometric transients, same filters would provide a more accurate measurement
of the brightness change, while different filters would provide a color constraint). The current
baseline survey strategy places no preference on whether pairs are obtained in the same or different
filters, which results in most pairs being in the same filter due to the additional time penalty in
changing the filter. The candidate baseline survey has just under 11,000 filter changes over the 
ten years of the survey, with an average of 3 filter changes per night; about 7000 of these changes 
can be attributed to the WFD portion of the survey.
\item The current strategy for the main survey which obtains two visits per night could be 
modified to obtain a single visit, or more than two visits, per night. 
\item The current baseline survey strategy assumes that a visit is composed of two 15-second
exposures, the so-called snaps. While 2x15 sec visits enable search for very rapid variability,
and help with cosmic ray rejection, there are compelling technical arguments -- including observing efficiency -- 
to adopt single-exposure 30 second visits.  The effect of varying the exposure time and number of snaps in a 
visit can be calculated from details available in the LSST overview paper, but a summary of the effect on the $5\sigma$ limiting magnitude 
is shown in Table~\ref{m5.tab}. Arguments for or against retaining the 2x15 sec visits 
would be very useful in further optimization of the main survey strategy. 
The impact of read-out noise and other parameters on limiting depth is quantitatively
discussed in the LSST Overview paper (see Section 3.2.1). 
\end{itemize}
White papers addressing all or some of these optimization efforts are strongly encouraged.

\begin{table}[htbp]
\centering
\caption{Changes in $5\sigma$ limiting magnitude when varying the exposure time and number of snaps per visit.}
\begin{tabular}{lrrrrrr}
\hline
{$N_{exp}$ x $T_{exp}$} &      u &      g &      r &      i &      z &      y \\
\hline
2x15 &  0.000 &  0.000 &  0.000 &  0.000 &  0.000 &  0.000 \\
1x20 & -0.129 & -0.186 & -0.202 & -0.207 & -0.211 & -0.214 \\
1x30 &  0.216 &  0.078 &  0.044 &  0.033 &  0.023 &  0.017 \\
1x40 &  0.447 &  0.259 &  0.214 &  0.199 &  0.186 &  0.179 \\
2x30 &  0.601 &  0.460 &  0.425 &  0.413 &  0.402 &  0.396 \\
\hline
\end{tabular}\\ \vskip 0.05in
\label{m5.tab}
\end{table}


\subsection{``Deep Drilling'' mini surveys} 

The Deep Drilling (DD) mini surveys are a set of mini surveys, each consisting of a field (a single pointing) 
that is observed in extended sequences, potentially with specified dithering patterns.
In the first call for white papers in 2011 on survey strategy optimization, the Project received 
8 white papers from the community\footnote{These white papers are available from 
https://project.lsst.org/content/whitepapers32012}. 
These white papers described excellent early ideas that helped design the current baseline 
cadence simulation; however, to receive full consideration, these ideas will have to be 
resubmitted as new white papers. Many of these early proposals include specific 
filter combinations to ensure that near-simultaneous color information is available for 
variable and transient objects. While science programs suggested in DD proposals are ideally well 
matched to the size of LSST field of view (9.6 deg$^2$), it is plausible that some
programs may require several fields. 

Four of the LSST DD mini survey field locations have been selected and announced\footnote{Please see http://ls.st/bki} (Elais S1, 
XMM-LSS, Extended Chandra Deep Field-South, and COSMOS). It is guaranteed that they 
will be observed with deeper coverage and more frequent temporal sampling than the main 
survey fields, but details are still open and no minimums have been set. Thus four mini surveys,
representing these four DD fields, are required. White papers detailing additional DD mini surveys and their
requirements are solicited with this call (a plausible but not prescriptive range is 5-10).

The observing sequences and coadded depths for these DD mini surveys are not yet decided. The current baseline
cadence includes sequences of $grizy$ observations during bright and gray time, and sequences of $u$ 
band observations during dark time. The large number of filter changes in the bright time sequences
are inefficient and the large gap in multi-color sampling during dark time is likely problematic 
for variable and transient characterization. White papers addressing improved cadences 
in the DD mini surveys are desirable.


\subsection{Galactic plane mini survey \label{sec:GP}}

The baseline main survey excludes observations at low Galactic latitudes, where the high 
stellar density leads to a confusion limit at much brighter magnitudes than those attained 
at high Galactic latitudes. Assuming median seeing, this confusion limit corresponds to a
source density of about 1-2 million per deg$^2$. The current boundary of this ``Galactic
confusion zone'' starts at $|b|=10^\circ$ towards $l=0^\circ$ and linearly drops to $b=0^\circ$
at $l=90^\circ$ and $l=270^\circ$. Along this boundary, the confusion limit is reached at a
depth of $i \sim 26$; therefore, the useful coadded depth is at least 1-2 magnitudes 
shallower than for the main survey and the total number of visits in this region is thus fewer. 
Within this boundary, the fraction of galaxies is only a few percent (due to flux extinction 
by interstellar dust) and an assumption that all sources are stars works quite well. 
Crowding is not expected to significantly impact the quality of data products derived from difference images
({\it i.e.} Prompt data products). 

While the confusion limit is relevant for the coadded depth and is the reason the current version of this mini survey
contains so few total visits (30 per field in each of the six filters), time-domain studies using photometry from 
single images could still benefit from additional visits in this region. Science driven input is needed for both ``static science'' 
and time domain surveys. White papers exploring this science case are encouraged.

As guidance, stellar count simulations with the TRILEGAL code (Girardi et al. 2005, 
arXiv:astro-ph/050404) show that to the depth of $r=27.5$ there are about 4 billion
stars in the main survey area, with another 13 billion stars in the Galactic confusion 
zone. Of the latter, about 6 billion are brighter than $r=24.5$.  

The footprint in the current baseline survey strategy extends to the far north along the Galactic
plane, to the region that can only be observed at relatively large airmass from the LSST
site at Cerro Pach\'on ($X>1.4$ at $\delta  = +15^\circ$). Originally, this extension was designed 
to extend longitudinal coverage of the Galactic plane with Galactic structure studies in mind. 
With the advent of other surveys (e.g., Pan-STARRS and DECAPS\footnote{See http://decaps.skymaps.info}), 
the reasons for obtaining these less efficient observations (due to unavoidable high airmass) are less compelling. 
Unless a strong case is made in submitted white papers, the Project is likely to limit the 
coverage of the Galactic plane to $\delta < +2^\circ$. 


\subsection{Southern Celestial pole mini survey}

Due to its southern declination limit ($\delta > -62^\circ$), the main survey misses a large fraction
of both the Magellanic Clouds. To allow coverage of the Large and Small Magellanic Clouds, the 
baseline survey strategy uses relaxed limits on airmass and seeing for the $\sim$2,000 deg$^2$ region 
around the South Celestial Pole, but with fewer observations than for the main survey. 

Detailed optimization of this strategy has not been done yet. Given recent informative observations 
obtained by the SMASH survey (Nidever et al. 2017, Astronomical Journal, 154, 199), as well as calibration 
and legacy aspects of this mini survey, a white paper with more detailed cadence prescriptions
(e.g., is it necessary to extend the coverage all the way to the South Celestial pole?) would greatly 
inform further cadence optimization. 


\subsection{Northern Ecliptic spur mini survey}

The main survey footprint provides most of LSST's power for detecting Near Earth Objects (NEO) and 
TransNeptunian Objects (TNOs) and naturally incorporates the southern half of the Ecliptic within its 
18,000 deg$^2$ sky area. The additional coverage of a crescent reaching to $+10$ degrees of the Northern Ecliptic 
plane in the North Ecliptic Spur (NES) mini survey provides observations of small bodies, particular TNOs, 
throughout the full range of ecliptic longitude. The baseline survey strategy covers this region using the 
$griz$ filters only, with about 300 visits per field and a cadence generally similar to that of the main survey
(but with more relaxed limits on airmass and seeing due to the northern location of the fields). With fewer visits per field,
it is not possible to maintain a complete main survey time sampling over the full ten years of the survey. Previous
simulations ended observations in the NES by approximately year seven; the current baseline simulation runs 
observations over the full ten years at the cost of a lower sampling rate (the mean inter-night gap falls from approximately
every 3 nights in minion\_1016 to approximately every 5 nights in the current baseline, baseline2018a).

A more detailed  and robust science-driven justification addressing both outer Solar System (e.g., sampling
of the full longitudinal distribution of TNOs) and inner Solar System (e.g., light curve sampling for main-belt 
and NEO asteroids) populations is needed to maintain and optimize the observing strategy for this
mini survey.  


\subsection{Twilight survey \label{sec:twilight}} 

LSST's short read-out time (2 sec) enables efficient taking of short-exposure images during twilight time 
that would otherwise go unused. Science drivers and technical details are discussed in Section 10.3 in the 
COSEP (arXiv:1708.04058); the former include a bright star survey for Galactic
science, obtaining light curves for nearby supernovae, and observations of near-Earth asteroids towards
so-called ``sweet spots'' (on the Ecliptic, at Solar elongations of $\sim60^\circ$). This cadence has not 
been simulated yet. 

The Project plans to simulate twilight observing in late 2018.
Assuming exposure times of 1 second (the stretch goal from the LSST Science Requirements Document, 
which is already met and corresponds to the Camera baseline requirement), the saturation limit would be
about 3 mag brighter than for 15-sec exposures; in the $r$ band from about $r=16$ to $r=13$,
with about 1.5 mag loss of limiting depth. In morning twilight, the improvement in dynamic range of 1.5 mag would gradually
diminish and eventually dynamic range would vanish as the sky brightness increases. About 20-30 minutes
of additional observing time could be utilized during twilight before the dynamic brightness range becomes
too small. Assuming 7-sec visits (1 sec exposure + 1 sec for shutter + 5 sec for read and slew), about 
2,000 deg$^2$ of sky could be imaged in 25 minutes in a given filter. Alternatively, over 
350 exposures (1 sec + 1 sec + 2 sec) of the same field could be obtained instead.  Detailed strategies
should consider limitations on the number of filter changes (see Appendix~\ref{sec:HW}). 
White papers addressing the science justification and strategies for using twilight time are desired.

The shortest exposure stretch goal in the Camera baseline requirements is set to 0.1 sec. Science-driven
studies that would advocate pursuing this goal would be a welcome contribution to further system
optimization; twilight observing could especially benefit from shorter exposures. 
 

\section{Constraints on survey strategy imposed by the LSST system} 

\subsection{Hardware and software constraints \label{sec:HW}}

\leftline{\bf Hardware constraints}

Several hard constraints prevent observations in certain alt-az directions. The telescope altitude limit
prevents observations at altitudes below 20$^\circ$.  As a consequence of the alt-az mount, there
is also a zenith exclusion zone with a radius of 3.5$^\circ$. 

The LSST telescope mount uses direct drive motors and there should not be any mechanical limits 
on slewing from the mount.  However, there are observing efficiency considerations: the minimum
slew time (as soon as the telescope moves at all) is 3 seconds, due to readout (2s) plus settle time requirements (1s). 
Otherwise, the slew time depends on slew distance. Approximately\footnote{These expressions are 
approximate linear fits to the precise model behavior used in OpSim.}, 
in the azimuth direction, 
\begin{eqnarray}
             t_{slew}^{Az} = 0.66 \, {\rm sec/deg} * \delta Az ({\rm deg}) + C^{Az} \\
             t_{slew}^{Az} {\rm min} = 3\, {\rm sec}
\end{eqnarray} 
where $C^{Az} = -2$ sec (this is negative because of dome crawl; however, the minimum
slew time is still 3 seconds due to readout and telescope settle time). For slewing in altitude
\begin{equation}
             t_{slew}^{Alt} = 0.57 \, {\rm sec/deg} * \delta Alt ({\rm deg}) + C^{Alt} ,  
\end{equation} 
where $C^{Alt} = 3$ sec for slews below 9$^\circ$ and $C^{Alt} = 37$ sec for longer slews (because 
of the need to recompute optics corrections for slews larger than 9 degrees in altitude). 
The dome is assumed to crawl in the azimuth direction, but not in altitude. 
In the baseline simulated survey, about 2\% of slews move
in altitude more than 9$^\circ$. 

The shortest exposure time is assumed to be 1 second, with the shortest exposure stretch goal in the Camera 
baseline requirements set to 0.1 sec. For exposures shorter than 
about 10 sec, the seeing due to atmospheric turbulence may be harder to characterize (the profiles are more
irregular) than for longer exposures, and moving objects may trail in long exposures (for more
details see Section 5.1.4 in Jones et al. 2018; Icarus 303, 181). Short exposures will have a low
observing efficiency due to the finite read-out time (2 sec). % and the shutter open/close time (1 sec). 

There are important constraints on the filter exchange strategy. As the system is not yet completely 
built and characterized, the following represents current understanding of the limitations on filter 
loads (swaps in and out of the carousel, done during the daytime) and filter changes (swaps within
the carousel, done between exposures within a night), based on the design and on 
engineering judgement. As such, some of the details should be considered preliminary and subject 
to change. Expanded ranges could be possible if there are strong scientific motivations along with
sufficient resources during operations. 
\begin{itemize}
\item LSST has 6 filters ($ugrizy$), of which 5 can be loaded into the filter exchange carousel. Only 5 out of
the 6 filters will be available on any given night. 
\item Filter load operations (swapping a filter into or out of the carousel, to change the set of five available filters) 
	will be done during daytime. The system is designed for 3000 loads over its lifetime. 
\item The currently implemented filter load strategy is to replace during the day one of the $z$ or $y$ 
	by the $u$ band at the start of dark time, when the lunar phase reaches $20\%$. The 
	process is reversed at the end of dark time when the lunar phase is above the same threshold.
\item During a given observing night, the system could support as many filter changes (changes within a night, 
	involving the 5 filters already loaded in the carousel) as desired, without any practical limitation beyond 
	the two-minute change interval (which consists of 90 seconds for the exchange plus up to 30 seconds 
	to put the camera into the required orientation). 
\item The carousel filter change mechanism is designed to undergo a total of 100,000 filter changes over its lifetime 
	(an average of about 17 changes per night of the survey, after accounting for necessary calibration activities). 
\item Each individual filter is designed to support up to 30,000 changes over its lifetime.
\item A maintenance cycle to the filter exchange mechanism is anticipated, and this would nominally occur after 10,000 filter changes or one year, whichever is reached first. 
\end{itemize} 


\leftline{\bf Software constraints} 

The LSST Science Requirements Document specifies that ``As a general principle, the measurement errors
for fundamental quantities, such as astrometry, photometry and image size, should not be dominated by 
algorithmic performance.'' Data products that LSST will produce are described in LSST Data Products
Definition Document (ls.st/dpdd) and more algorithmic detail is provided in LSST Data Management 
Science Pipelines Design document (ls.st/ldm-151). 

The Project will not take formal responsibility for specialized data reduction algorithms 
needed to process data, including that taken in ``non-standard'' modes; detailed discussion is 
available in the Data Management and LSST Special Programs document (ls.st/dmtn-065) and should
be perused when proposing non-standard observing sequences. In addition, we strongly recommend 
that white paper authors consult Sections 5 and 6 in the \href{http://ls.st/dpdd}{LSST Data Products Definition Document}. If 
a proposed dataset will require special processing, a plan to obtain necessary software and compute resources 
must be provided in the white paper. 

There is an additional caveat regarding crowded field processing (see also Appendix~\ref{sec:GP}). 
A fraction of LSST imaging will cover areas of high object (mostly stellar) density, such as the 
Galactic plane, the Large and Small Magellanic Clouds, and a number of globular clusters (among 
others). LSST image processing and measurement software, although primarily designed to operate 
in non-crowded regions, is expected to perform well in areas of crowding. The current LSST applications 
development plan envisions making the deblender aware of Galactic longitude and latitude, and 
permitting it to use that information as a prior when deciding how to deblend objects. While not 
guaranteed to reach the accuracy or completeness of purpose-built crowded field photometry codes, 
we expect this approach will yield acceptable results even in areas of moderately high crowding.

The above discussion only pertains to processing of direct images. Crowding is not expected to 
significantly impact the quality of data products derived from difference images (i.e., Prompt 
products).


\subsection{Observing efficiency constraints} 

The LSST Science Requirements Document  ``...assumes a nominal 10-year duration with about 90\% 
of the observing time allocated for the main LSST survey.'', and thus 10\% of observing time is left for 
all other programs. However, if the system performs better than expected, or if science priorities 
change over time, it is conceivable that the 90\% could be modified and become as low as perhaps 80\%, 
with the observing time for other programs thus doubled. At this time, details are TBD but the Project
is developing flexible scheduling procedures to enable such modifications. In the current baseline
survey strategy, the main survey takes about 86\% of the time, with the remainder taken by the candidate mini surveys. 

We note that the uncertainty in our system performance estimates due to weather and solar activity is about 10\%.
In addition, the system has not been built yet and many hardware performance parameters are
still taken at their design values.  

Sustained observing, such as the main survey, with exposures much shorter than standard visits will result
in diminished observing efficiency. Given total visit exposure time $t_{vis}$ (30 sec for standard
visits), with two exposures/readouts (snaps) per visit, and assuming a slew and settle time of 5 sec 
(also including the second readout), the observing efficiency can be computed as 
\begin{equation}
     \epsilon = \left( {t_{vis} \over t_{vis} + 9 \, \mathrm{sec}}\right).
\end{equation}
To maintain efficiency losses below $\sim$30\% (i.e., at least below the limit set by the weather patterns),
and to minimize the read noise impact, $t_{vis} > 20$ seconds is required for sustained observing. 

Variations in exposure time for the main survey affect not only the limiting depth, but also the total number of 
acquired visits and revisit time because the total observing time is finite. For more detailed 
discussion of these tradeoffs, please peruse Section 2.2.2 in the Overview paper. 

The number of filter changes (2 minutes per change) and size of slews also affects efficiency. 


\subsection{Limiting depth and uncertainty estimates}  

Methods for estimating individual image depth ($5\sigma$ point source magnitude limits) for a given 
exposure time and other observing parameters
are discussed in detail in Section 3.2 in the LSST Overview paper (see \ref{sec:pubs}). Tradeoffs between 
exposure time per visit, single-visit depth, the mean revisit time, and the total number of visits,
as well as justification for the adopted standard exposure time of 30 sec, are discussed in Section 
2.2 of the same paper. The improvement in measurement uncertainties as the survey progresses,
as a function of time $t$, can be approximately summarized as follows. 

The co-added depth (the 5$\sigma$ depth for point sources), $m_5^{\rm co-add}$, scales 
with time as (see eq.~6 in the overview paper)  
\begin{equation} 
         m_5^{\rm co-add}  = m_5^{\rm co-add, Final}  + 1.25 \, \log_{10}\left({t \over 10 \, {\rm yrs}}\right) 
\end{equation} 
where $m_5^{\rm co-add, Final}$ is the target depth achieved with the 10-year survey. With
airmass and other losses taken into account, $m_5^{\rm co-add, Final}=27.2$ for the $r$ band
in the baseline simulated survey. 

The photometric errors (inverse signal-to-noise ratio) at the faint limit of the so-called 
``gold'' galaxy sample (4 billion galaxies with $i<25.3$ which will be used for cosmological
programs, see Section 3.7.2 in the \href{https://www.lsst.org/scientists/scibook}{LSST Science Book}\footnote{The 
\href{https://www.lsst.org/scientists/scibook}{LSST Science Book} 
is a wide-ranging, detailed description of science cases enabled by LSST. Written in 2009 by the LSST Science Collaborations, it is available online
at \href{https://www.lsst.org/scientists/scibook}{https://www.lsst.org/scientists/scibook}.}), is computed from (see eq. 5 and Table 1 
in the overview paper):
\begin{equation} 
                \sigma_{i=25} = 0.04 \, \left({t \over 10 \, {\rm yrs}}\right)^{-1/2} {\rm mag.}
\end{equation}

The volume of the 5-dimensional color error space per source with $i=25$,  which controls the ability 
to classify sources using colors (including photometric redshift estimates for galaxies and star/quasar
separation, for example) is computed assuming uncorrelated color errors, as proportional
to $\sigma^5_{i=25}$, and normalized by the value corresponding to the 10-year survey. 

The trigonometric parallax accuracy for a point source with $r$=24 (see section 3.3.3 in the 
overview paper) scales with time as 
\begin{equation}
        \sigma_\pi = 3.0 \,  \left({t \over 10 \, {\rm yrs}}\right)^{-1/2}  \,\,  {\rm mas.} 
\end{equation}

The proper motion accuracy for a point source with $r$=24 (see section 3.2.3 in the overview paper)
scales with time as 
\begin{equation}
        \sigma_\mu = 1.0 \,  \left({t \over 10 \, {\rm yrs}}\right)^{-3/2}   \,\, {\rm mas/yr.} 
\end{equation}
Note the very strong dependence of  $\sigma_\mu$ on time ($t^{-1/2}$ comes from the
increase in the square root of the number of visits, analogously to $\sigma_\pi$, and 
an additional $t^{-1}$ from the linear increase in temporal baseline).  In both expressions,
the number of visits is assumed proportional to time, with a value of 825 corresponding to the 
main wide-fast-deep 10-year survey. 

The behavior of these quantities as a function of time is summarized in Table 2. While 
the co-added depth and $\sigma$($i$=25) rapidly improve during the first few years, 
several important quantities continue to show marked improvement between survey 
years 8 and 10: most notably, the color error volume per source for faint sources ($i=25$) 
shrinks by a factor of 1.7.  Substantial improvement is also seen for proper motions, 
with errors larger by 40\%, after 8 years than at the end of the 10-year survey.  


\begin{table}[h!]
\caption{Various science metrics as functions of survey duration.}
\begin{tabular}{|l|r|r|r|r|r|r|}
\hline     
          Quantity                          &     Year 1   &    Y3  &     Y5  &     Y8   &     Year 10   \\
\hline  
    $r_5$ coadd$^a$                   &       26.0    &      26.5   &      26.8    &      27.1    &          27.2     \\
    $\sigma$($i$=25)$^b$         &     0.12    &     0.07    &      0.06    &    0.05      &        0.04        \\     
    color vol.$^c$                        &       316     &       20     &        6      &    1.7        &           1       \\
     \# of visits$^d$                    &          83     &     248     &      412     &    660      &          825      \\  
    $\sigma_\pi$ ($r$=24)$^e$   &        9.5     &      5.5     &        4.2    &     3.3       &          3.0      \\ 
    $\sigma_\mu$ ($r$=24) $^f$  &  32   &      6.1    &     2.8   &     1.4   &     1.0     \\
\hline                         
\end{tabular}
\\ \vskip 0.05in
$^a$ The co-added depth in the $r$ band (AB, 5$\sigma$; point sources).  \\
$^b$ The photometric error for a point source with $i=25$. \\
$^c$ The volume of the 5-dimensional color error space, normalized by the final value. \\
$^d$ The number of visits per sky position (summed over all bands). \\
$^e$ The trigonometric parallax accuracy for a point source with $r$=24 (milliarcsec). \\ 
$^f$  The proper motion accuracy for a point source with $r$=24 (milliarcsec/yr).  \\
%\vskip 0.2in          
\end{table}


\newpage
\section{Supplementary materials \label{sec:supp}} 
\label{append:supplemental}

\subsection{Useful publications and websites \label{sec:pubs}}

{\it Note that various websites and documents references here are still in development (if not an official LSST document under change control).}

The \href{https://www.lsst.org/content/lsst-science-drivers-reference-design-and-anticipated-data-products}{LSST Overview paper} provides a short summary of the four primary science drivers, as well as the expected performance of LSST in terms of throughputs, and calibration. It also discusses high-level survey constraints and tradeoffs.  Last updated in 2018 (but a living document that will have future incremental updates), this is available as \url{http://ls.st/lop}

The \href{https://www.lsst.org/scientists/scibook}{LSST Science Book} tackles an in-depth view of a broad range of LSST science goals. Written in 2009 by the LSST Science Collaborations, each chapter focuses on investigations relevant to a different area of astronomy. Available at \url{https://www.lsst.org/scientists/scibook}

The \href{https://github.com/LSSTScienceCollaborations/ObservingStrategy}{Community Observing Strategy Evaluation Paper} (COSEP) is a community-driven paper describing a wide variety of science cases and their implications for survey strategy. This paper is primarily aimed at helping define the main wide-fast-deep survey. Last updated in 2017, the analysis in the COSEP is currently based on older simulations (including a reference simulation similar to the current baseline, but with significant differences in the detailed observations). The COSEP will continue to be a living document, with periodic updates as provided by authors. This is available as \url{http://ls.st/3y1}

The \href{http:/ls.st/srd}{LSST Science Requirements Document} (SRD) describes the official requirements for LSST science deliverables. Section 3.4 is the most relevant for survey strategy, although other sections are relevant for telescope and camera performance such as throughputs and readout time. This document is under change control, and is not subject to update. Available as \url{http://ls.st/srd}

The \href{http://ls.st/dpdd}{LSST Data Products Definition Document} (DPDD) describes the data products that LSST will provide, with some high-level background on how they will be produced. If you want to know what will be contained in various catalogs, this is a good place to look. Available as \url{http://ls.st/dpdd}

The \href{http://ls.st/ldm-151}{LSST Data Management Science Pipelines Design} (LDM-151) document describes the LSST data management processing pipelines. This provides details of how and when images will be processed and catalogs will be generated, including information on the algorithms used in each processing stage. If you want to know more about the details of a value in an output catalog and how it will be calculated, this is the place to look. This document will be periodically updated as processing algorithms are improved. Available as \url{http://ls.st/ldm-151}

%The \href{http://ls.st/lse-61}{LSST Data Management System Requirements} (DMSR) document details the official requirements for LSST's data management processing. This is a higher level of detail than the DPDD on the characteristics and specifications of individual data products but does not describe how these data products are created. \url{http://ls.st/lse-61}

Documentation about the LSST Operations Simulator (OpSim), which consists of the Simulated Observatory Control System (SOCS) and 
Scheduler is available at \url{https://lsst-sims.github.io/sims_ocs/}.   
Documentation about the LSST Metrics Analysis Framework (MAF) is available at \url{http://sims-maf.lsst.io}.  Documentation about running
OpSim and MAF via docker is available at \url{http://ls.st/40m}.

The git repository (\href{https://github.com/lsst-pst/survey_strategy/}{https://github.com/lsst-pst/survey\_strategy/}) hosting the source of this document also holds additional information relevant to the call for white papers, such as a summary of observing constraints. The contents of the repository are described in the \href{https://github.com/lsst-pst/survey_strategy/blob/master/README.md}{README}. 

The baseline2018a simulated survey document describes the recent, post-COSEP, features enabled in OpSim, as well 
as characteristics of the updated baseline simulated survey ('baseline2018a'). This document is available at \url{http://ls.st/Document-28453}.

The `next candidate reference run' ({\it i.e.} the run that is likely to become baseline2018b) simulated survey document describes 
recent updates in OpSim to increase simulation fidelity. These updates include simulating dome crawl, which increases the efficiency 
of observing by about 3\%, and updating the optics corrections time delay to the currently expected requirements. This document is available at 
\url{http://ls.st/Document-28715}.

A summary of the alternate survey strategy simulations produced as examples for the call for white papers, listed below in Section~\ref{sec:surveys}
is available at \url{http://ls.st/Document-28716}. Additional plots and metric comparison are available in the survey\_strategy github repo, at \url{http://ls.st/blm}.

The full outputs of MAF analyses for the 2018a baseline survey, as well as the example runs generated for the call for white papers, are available at \url{http://astro-lsst-01.astro.washington.edu:8080}.  

A short description of the current Deep Drilling mini surveys and links to further materials (including white papers submitted in response to the 2011 call for input on the DD strategy) are available on the LSST website at \href{https://www.lsst.org/scientists/survey-design/ddf}{http://ls.st/57q}. Additional posts on LSST Community can be found by searching for \href{https://community.lsst.org/search?q=deep%20drilling}{`deep drilling'}. 

Additional information is available in the following presentations:
\begin{itemize}
\item ``Overview of the LSST Observing Strategy'' (Nov 16, 2015): \url{http://ls.st/4yh}
\item ``The LSST Deep-Drilling Fields: White Papers and Science Council Selected Fields'' (Aug 15, 2016): \url{http://ls.st/wzy}
\item ``Observing Strategy White Paper Status Report'' (Mar 5, 2017): \url{http://ls.st/zj2}
\item ``LSST Plans for Cadence Optimization'' (May 30, 2017): \url{http://ls.st/ot2}
\item ``Special Programs'' (Aug 15, 2017): \url{http://ls.st/10o}
\end{itemize}


\subsection{Communication about the LSST survey strategy} 

The LSST Project Scientist (\v{Z}eljko Ivezi\'{c}, e-mail: ivezic at astro.washington.edu) is formally responsible for survey strategy optimization efforts and is the formal liaison between the community and the LSST Scheduler and Operations Simulation teams.

The LSST Science Advisory Committee (SAC) (\url{https://project.lsst.org/groups/sac}) is charged with collecting and delivering various community input to the Project. Strategic and political issues about the LSST survey strategy should be communicated via the SAC (chair: Michael Strauss, strauss at astro.princeton.edu).

In addition to this call for white papers, the Community Observing Strategy Evaluation Paper\footnote{A living document (available
as \url{http://ls.st/9fw}) and a community project coordinated at https://github.com/LSSTScienceCollaborations/ObservingStrategy}, COSEP, provides a coordinated mechanism for providing scientific input about survey strategy. LSST science collaborations are also official channels for communication with the LSST project --- a Data Management 
liaison\footnote{For the list of Science Collaboration contacts and corresponding Data Management Liaisons, please see \url{https://www.lsstcorporation.org/science-collaborations}.}  is assigned to each Science Collaboration to answer specific questions about data products generated by the project.

An open, searchable resource for asking questions not addressed in this document is available on \href{http://community.lsst.org}{LSST Community}, in the Science category, SurveyStrategy subcategory. Team members will monitor and respond in a timely manner to questions posted there. Please go to \url{http://community.lsst.org/c/sci/survey-strategy}

There is a mailing list available (email: lsst-survey-strategy at lists.lsst.org) to contact the survey strategy team in case of specific 
questions and/or concerns. Messages posted to the mailing list are broadcasted to the survey strategy team and archived. 
The same list will be used for white paper submission (see Section~\ref{sec:guidelines}). 

Throughout the survey strategy design process there will be open community meetings. The first of these will be 
at the LSST 2018 Project and Community Workshop in Tucson, AZ (August 13-17, 2018). More information on this 
meeting, including the agenda, is available at \url{https://project.lsst.org/meetings/lsst2018/}; note that the registration
deadline is July 9, 2018. Another highly relevant
meeting/workshop will be organized by LSST Corporation and the Simons Center for Computational Astrophysics
at the Flatiron Institute in Manhattan, NY (September 17-19, 2018). To apply for this workshop, please fill out the 
application form at \url{http://ls.st/cwg} by July 20, 2018. Additionally, the Project will organize a dedicated session at the
LSST 2019 Project and Community Workshop to discuss submitted white papers and follow-up
deliberations and work. We anticipate more similar meetings before the formal start of LSST Operations. 

The Project teams cannot support individuals or groups wishing to run the Operations Simulator themselves. 
The Project will provide Docker images (as well as the source code) and corresponding documentation on running 
OpSim; however, the Project does not have resources to provide help desk facilities on this topic. 


\subsection{Additional Simulated Survey Strategies \label{sec:surveys}}

With this call for white papers, we are providing a set of
simulated surveys for the community to use as test cases. These simulated survey strategies will include (as of June 30, 2018): 
\begin{itemize}
\item The baseline survey strategy, with the main survey having a 18,000 deg$^2$ footprint, the North Ecliptic Spur, South Celestial Pole, Galactic Plane and Deep Drilling mini surveys operating in the current example baseline survey strategy manner (i.e., no ``rolling''). 
\item A slight variation on the baseline survey strategy, extending the main survey by about 1.5 degrees further north and south to improve performance on 
the SRD requirements about footprint and number of visits. 
\item A survey with a much larger main survey footprint (27,000 deg$^2$), defined by airmass limit $X<1.5$ and declination limits 
$-78^\circ< \delta < +18^\circ$, and including only the four-field Deep Drilling mini survey (and no other mini surveys).
\item A survey strategy allowing more overall visits by making each visit shorter and without snaps: each visit being 20 sec long with a single exposure per visit in $g$, $r$, $i$, $z$ and $y$, and 40 sec visits in the $u$ band.  
\item A survey similar to the current baseline, with the Galactic Plane at the main survey cadence (and the northern portion of the Galactic plane mini survey with $\delta>2^\circ$ removed).
\item A survey similar to the current baseline, but only observing with single visits per night.
\item A survey similar to the current baseline, but adding more Deep Drilling mini surveys to increase the number of DD fields to 9 (from 5). 
\item A rolling cadence survey with two declination bands that divide the WFD survey area in half and alternate on/off every other year. The non-rolling WFD proposal was completely removed from this survey.
\item A rolling cadence survey with two declination bands that divide the WFD survey area in half and alternate on/off every other year. The non-rolling WFD proposal was included in this survey at the 25$\%$ level, as a low-level, on-going, background. 
\item A rolling cadence survey that observed the entire WFD area for the first and last two years of the survey. During the six years in between, the WFD area was divided into three declination bands observed every three years (i.e. each band is observed twice).
\end{itemize}

A summary of these simulations is available at \url{http://ls.st/Document-28716}. 
Additional plots and metric comparison are available in the survey\_strategy github repo, at \url{http://ls.st/blm}.
The full outputs of MAF analyses are available at \url{http://astro-lsst-01.astro.washington.edu:8080}.  

We anticipate that additional sample survey strategies will follow during the remainder of 2018, which may include changes reflecting updates to the scheduling software.

\end{document} 



===== DISSEMINATION =====  

* details in Jason Kalirai's email from May 14 

0) in addition to sending announcement to LSST-world exploder, 

1) write a short article for AAS News Digest  (with links to docs in LSST Docushare) 

2) also send it to email exploders: STScI, NOAO, IPAC, SDSS and the WFIRST SITs.
 
3) also email it to department heads to forward to their faculty, postdocs, and students

4) post on social media: Facebook Astronomers and LSST, LSST tweeter account, 
 