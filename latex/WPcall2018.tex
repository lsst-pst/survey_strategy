\documentclass[DM,lsstdraft,toc,usenatbib]{lsstdoc}

% Package imports go here
\usepackage{amsmath}	% Advanced maths commands
\usepackage{amssymb}
\usepackage{gensymb}  % degree symbol 
\usepackage{natbib}  % bibliography
\usepackage{cprotect} 
% Local commands go here

%% Journal abbreviations
%\bibliographystyle{aasjournal}

\title[Call for LSST Cadence White Papers]{Call for White Papers on LSST Cadence Modifications} 

\author{\v{Z}eljko Ivezi\'{c}, Lynne Jones, Tiago Ribiera, 
             \\  the LSST Project Science Team, 
              \\ and  the LSST Science Advisory Committee} 

\setDocRef{Document-XXX}
\date{\today}
\setDocRevision{v0.1}
\setDocStatus{draft}
\setDocAbstract{%
The LSST community is invited to play a key role in the refinement of LSST’s Observing Strategy 
by submitting white papers that will describe proposed modifications of the current baseline
cadence.
}

% Change history defined here. Will be inserted into
% correct place with \maketitle
% OLDEST FIRST: VERSION, DATE, DESCRIPTION, OWNER NAME
\setDocChangeRecord{%
\addtohist{1}{2018-06-30}{First released version.}{\v{Z}eljko Ivezi\'{c}}
}

\begin{document}

% Create the title page
% Table of contents will be added automatically if "toc" class option
% is used.
\maketitle

\section{Introduction} 

General context... evolving science, cadence...

The Large Synoptic Survey Telescope is designed to provide an unprecedented optical 
imaging dataset that will support investigations of our Solar System, Galaxy and Universe, 
across half the sky and over ten years of repeated observation. LSST is constructing a 
flexible scheduling system that can respond to the unexpected 
and be re-optimized. A basic implementation of LSST's 10-year survey (simulations of 
the observing strategy or ``cadence'') can deliver on a wide range of science. 
However, exactly how the LSST observations will be taken is not yet finalized. Indeed, 
it is anticipated tha the observing strategy will continue to be refined and optimized 
during operations.

The Community is playing a key role in the refinement of LSST’s Observing Strategy by 
developing and analyzing metrics for simulated observing strategies.
An open github community\footnote{
https://github.com/LSSTScienceCollaborations/ObservingStrategy}
is where this work is being assembled. How the detailed performance of the anticipated 
science investigations is expected to depend on small changes to the LSST observing 
strategy is explored in a living dynamically-evolving community white paper (the first
version was published as arXiv:1708.04058 in August 2017). The main lessons 
learned from the first version are: 1) The Project should implement, analyze and optimize 
the rolling cadence idea (driven by supernovae, asteroids, short timescale variability),
and 2) The Project should execute a systematic effort to further improve the ultimate 
LSST cadence strategy (e.g. sky coverage optimization, u band depth, special surveys, 
Deep Drilling Fields).

Through the end of construction and commissioning, this community Observing Strategy 
White Paper will remain a living document that is the vehicle for the community to 
communicate to the LSST Project regarding the Wide-Fast-Deep and mini-survey observing 
strategies. The LSST Project Scientist, \v{Z}eljko Ivezi\'{c}, will synthesize and act on the 
results presented in this paper, with support from the Science Advisory Committee and 
Survey Strategy Committee. He is responsible for cadence optimization efforts and is the 
formal liaison between the community and the LSST Scheduler and Operations Simulation teams.




\subsection{Motivation for this white paper call}

We have tools, we are close to first light... 

Given the``living'' white paper, explain why we need more white papers...


\subsection{Timeline}

Deadline, what will happen when afterwards...

Produce, analyze and document a set of Observing Strategies and present to 
the SAC for a final strategy recommendation (in 2020) to begin the survey.



\section{How to submit a white paper?} 

We need to provide a tex template... 


\vskip 0.0in
\newpage
{\it Acknowledgments:} this document has greatly benefited from discussions between 
the LSST Project Science Team, the LSST Science Advisory Committee and Kem Cook, 
Phil Marshall, Steve Ridgway, Daniel Rothchild, Peter Yoachim and numerous other members 
of the LSST Science Collaborations. 

\appendix


\section{Examples of Current Open Cadence Questions} 

Summarize issues addressed in the living Observing Strategy White Paper, 
including ``The top 10 questions''... 


1) WFD 

different bands in pairs of visits?

dithering? 

rolling cadence properties (RA vs. Dec rolling) 

area vs. coverage tradeoff 





\section{Cadence Constraints Imposed by the LSST System} 


\section{Supplementary Materials} 

\subsection{Useful publications and documents}

\subsection{Useful websites}

\subsection{Useful slide collections}


\section{Communicating with LSST} 

The Observing Strategy white paper, and calls for DDF and mini-survey white papers, 
are the main mechanisms for providing scientific input about cadence. 
\v{Z}eljko Ivezi\'{c} (ivezic at astro.washington.edu) is the point of contact.

The LSST Science Advisory Committee (SAC) is charged with collecting and delivering 
community input to the Project. Strategic and political issues should be communicated 
via SAC (chair: Michael Strauss, strauss at astro.princeton.edu).

Join a science collaboration –- a Data Management liaison is assigned to each Science Collaboration.
Can utilize lsstc.slack.com.

Open and archived discussions with the team (especially Data Management and Education and 
Public Outreach) on community.lsst.org.

\end{document} 

 


